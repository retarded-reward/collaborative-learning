\documentclass[compress]{beamer}
\usepackage[
    title={Collaborative Learning},
    subtitle={Una soluzione di reinforcement learning applicata a un nodo router di messaggi},
    event={Progetto fine corso Machine Learning},
    author={DLP, MC, LF},
    longauthor={Daniele La Prova, Matteo Conti, Luca Falasca},
    email={},
    institute={ML 2023-2024},
    longinstitute={Universita' degli Studi di Roma Tor Vergata},
]{unislides}
\usepackage{graphicx} % Required for inserting images
\usepackage{minted}
\usepackage{algorithm}
\usepackage{hyperref}
\usepackage{adjustbox}

\begin{document}

\begin{frame}[plain]
    \titlepage
\end{frame}

\section*{Introduzione}
%TODO:

\subsection*{Contesto}
%TODO:

\subsection*{Obiettivi}
%TODO:

\section*{Metodologia}
%TODO:

\subsection*{Agente}
%TODO:

\subsubsection*{AgentFaçade}
%TODO:

\subsubsection*{AgentFactory}
%TODO:

\subsubsection*{Decisions}
%TODO:

\subsection*{Simulazione}
%TODO:

\subsubsection*{Nodo}
\begin{frame}{Nodo}
    E' l'elemento della rete che è in grado di ricevere e inviare messaggi sulla base della scelta dell'agente, è composto da:
    \begin{columns}
        \column{0.5\textwidth}
            \begin{minipage}[b]{1\textwidth}
                \begin{itemize}
                    \item Controller, che implementa la logica del nodo
                    \item AgentClient, che permette la comunicazione con l'agente
                    \item Queues, una o più code che accodano messaggi
                \end{itemize}
            \end{minipage}
        \column{0.5\textwidth}
            \begin{minipage}{1\textwidth}
                \begin{adjustbox}{margin=0cm 0cm 0.1cm 0.2cm, center} % left, bottom, right, top
                    \includegraphics[width=1\textwidth]{figs/node_layout_3queues.png}
                \end{adjustbox}
            \end{minipage}
    \end{columns}
\end{frame}

\subsubsection*{Nodo - Controller}
\begin{frame}{Nodo - Controller}
E' il componente che implementa la logica del nodo, in particolare:
\vspace{0.5cm}
    \begin{columns}
        \column{0.6\textwidth}
            \begin{minipage}[b]{1\textwidth}
                \begin{itemize}
                    \item Campiona lo stato visibile al nodo
                    \item Interroga l'agente per ottenere l'azione consegnandogli la reward dell'azione precedente
                    \item Attua l'azione ricevuta dall'agente
                    \item Calcola la reward per le azioni ricevute dall'agente
                \end{itemize}
            \end{minipage}
        \column{0.4\textwidth}
            \begin{minipage}{.9\textwidth}
                \begin{adjustbox}{margin=0.5cm 0cm 0.1cm 0.2cm, center} % left, bottom, right, top
                    \includegraphics[width=1\textwidth]{figs/control_loop.png}
                \end{adjustbox}
            \end{minipage}
    \end{columns}
\end{frame}
\subsubsection*{Nodo - Agent Client}
%TODO:

\subsubsection*{Nodo - Queue}
%TODO:

\subsubsection*{SrcNode}
%TODO:

\subsubsection*{Network}
%TODO:


\section*{Risultati}
%TODO:

\subsection*{Condizioni}

% TODO: una subsection per ogni scenario

\section*{Conclusioni}
%TODO:

\subsection*{What we have learned?}
%TODO:

\subsection*{What we could improve?}
%TODO:

\end{document}
