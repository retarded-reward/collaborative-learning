\documentclass[compress]{beamer}
\usepackage[
    title={Collaborative Learning},
    subtitle={Una soluzione di reinforcement learning applicata a un nodo router di messaggi},
    event={Progetto fine corso Machine Learning},
    author={DLP, MC, LF},
    longauthor={Daniele La Prova, Matteo Conti, Luca Falasca},
    email={},
    institute={ML 2023-2024},
    longinstitute={Universita' degli Studi di Roma Tor Vergata},
]{unislides}
\usepackage{graphicx} % Required for inserting images
\usepackage{minted}
\usepackage{algorithm}
\usepackage{hyperref}

\begin{document}

\begin{frame}[plain]
    \titlepage
\end{frame}

\section*{Introduzione}
%TODO:

\subsection*{Contesto}
%TODO:

\subsection*{Obiettivi}
%TODO:

\section*{Metodologia}
%TODO:

\subsection*{Agente}
%TODO:

\subsubsection*{AgentFaçade}
%TODO:

\subsubsection*{AgentFactory}
%TODO:

\subsubsection*{Decisions}
%TODO:


\subsection*{Simulazione}
%TODO:

\subsubsection*{Nodo}
%TODO:

\subsubsection*{Nodo - Controller}
%TODO:

\subsubsection*{Nodo - Agent Client}
%TODO:

\subsubsection*{Nodo - Queue}
%TODO:

\subsubsection*{SrcNode}
%TODO:

\subsubsection*{Network}
%TODO:


\section*{Risultati}
%TODO:

\subsection*{Condizioni}

% TODO: una subsection per ogni scenario

\section*{Conclusioni}
%TODO:

\subsection*{What we have learned?}
%TODO:

\subsection*{What we could improve?}
%TODO:

\end{document}
