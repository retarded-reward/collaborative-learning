\documentclass[conference]{IEEEtran}
\IEEEoverridecommandlockouts
% The preceding line is only needed to identify funding in the first footnote. If that is unneeded, please comment it out.
\usepackage{cite}
\usepackage{amsmath,amssymb,amsfonts}
\usepackage{algorithmic}
\usepackage{graphicx}
\usepackage{textcomp}
\usepackage{xcolor}
\usepackage{dblfloatfix}
\usepackage{hyperref}
\usepackage{minted}
\def\BibTeX{{\rm B\kern-.05em{\sc i\kern-.025em b}\kern-.08em
    T\kern-.1667em\lower.7ex\hbox{E}\kern-.125emX}}
\begin{document}

\title{Collaborative Learning\\ \large Una soluzione di reinforcement learning\\ applicata a un nodo router di messaggi\\
{\footnotesize Progetto Machine Learning}
}
\author{\IEEEauthorblockN{Matteo Conti}
\IEEEauthorblockA{\textit{matricola qui}} \\
\and
\IEEEauthorblockN{Daniele La Prova}
\IEEEauthorblockA{\textit{0320429}} \\
\and
\IEEEauthorblockN{Luca Falasca}
\IEEEauthorblockA{\textit{matricola qui}} \\
}

\newcommand{\code}[1]{\texttt{#1}}

\maketitle

\begin{abstract}
%TODO:
\end{abstract}

\section*{Introduzione}
%TODO:

\subsection*{Contesto}
%TODO:

\subsection*{Obiettivi}
%TODO:

\section*{Metodologia}
%TODO:

\subsection*{Agente}
%TODO:

\subsubsection*{AgentFaçade}
%TODO:

\subsubsection*{AgentFactory}
%TODO:

\subsubsection*{Decisions}
%TODO:


\subsection*{Simulazione}
%TODO:

\subsubsection*{Nodo}
%TODO:

\subsubsection*{Nodo - Controller}
%TODO:

\subsubsection*{Nodo - Agent Client}
%TODO:

\subsubsection*{Nodo - Queue}
%TODO:

\subsubsection*{SrcNode}
%TODO:

\subsubsection*{Network}
%TODO:


\section*{Risultati}
%TODO:

\subsection*{Condizioni}

% TODO: una subsection per ogni scenario

\section*{Conclusioni}
%TODO:

\subsection*{What we have learned?}
%TODO:

\subsection*{What we could improve?}
%TODO:

\end{document}